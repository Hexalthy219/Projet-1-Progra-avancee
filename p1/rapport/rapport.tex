\documentclass[a4paper, 11pt, oneside]{article}

\usepackage[utf8]{inputenc}
\usepackage[T1]{fontenc}
\usepackage[french]{babel}
\usepackage{array}
\usepackage{shortvrb}
\usepackage{listings}
\usepackage[fleqn]{amsmath}
\usepackage{amsfonts}
\usepackage{fullpage}
\usepackage{enumerate}
\usepackage{graphicx}             % import, scale, and rotate graphics
\usepackage{subfigure}            % group figures
\usepackage{alltt}
\usepackage{url}
\usepackage{indentfirst}
\usepackage{eurosym}
\usepackage{listings}
\usepackage{color}
\usepackage[table,xcdraw,dvipsnames]{xcolor}

% Change le nom par défaut des listing
\renewcommand{\lstlistingname}{Extrait de Code}

% Change la police des titres pour convenir à votre seul lecteur
\usepackage{sectsty}
\allsectionsfont{\sffamily\mdseries\upshape} 
% Idem pour la table des matière.
\usepackage[nottoc,notlof,notlot]{tocbibind} 
\usepackage[titles,subfigure]{tocloft} 
\renewcommand{\cftsecfont}{\rmfamily\mdseries\upshape}
\renewcommand{\cftsecpagefont}{\rmfamily\mdseries\upshape} 

\definecolor{mygray}{rgb}{0.5,0.5,0.5}
\newcommand{\coms}[1]{\textcolor{MidnightBlue}{#1}}

\lstset{
    language=C, % Utilisation du langage C
    commentstyle={\color{MidnightBlue}}, % Couleur des commentaires
    frame=single, % Entoure le code d'un joli cadre
    rulecolor=\color{black}, % Couleur de la ligne qui forme le cadre
    stringstyle=\color{RawSienna}, % Couleur des chaines de caractères
    numbers=left, % Ajoute une numérotation des lignes à gauche
    numbersep=5pt, % Distance entre les numérots de lignes et le code
    numberstyle=\tiny\color{mygray}, % Couleur des numéros de lignes
    basicstyle=\tt\footnotesize, 
    tabsize=3, % Largeur des tabulations par défaut
    keywordstyle=\tt\bf\footnotesize\color{Sepia}, % Style des mots-clés
    extendedchars=true, 
    captionpos=b, % sets the caption-position to bottom
    texcl=true, % Commentaires sur une ligne interprétés en Latex
    showstringspaces=false, % Ne montre pas les espace dans les chaines de caractères
    escapeinside={(>}{<)}, % Permet de mettre du latex entre des <( et )>.
    inputencoding=utf8,
    literate=
  {á}{{\'a}}1 {é}{{\'e}}1 {í}{{\'i}}1 {ó}{{\'o}}1 {ú}{{\'u}}1
  {Á}{{\'A}}1 {É}{{\'E}}1 {Í}{{\'I}}1 {Ó}{{\'O}}1 {Ú}{{\'U}}1
  {à}{{\`a}}1 {è}{{\`e}}1 {ì}{{\`i}}1 {ò}{{\`o}}1 {ù}{{\`u}}1
  {À}{{\`A}}1 {È}{{\`E}}1 {Ì}{{\`I}}1 {Ò}{{\`O}}1 {Ù}{{\`U}}1
  {ä}{{\"a}}1 {ë}{{\"e}}1 {ï}{{\"i}}1 {ö}{{\"o}}1 {ü}{{\"u}}1
  {Ä}{{\"A}}1 {Ë}{{\"E}}1 {Ï}{{\"I}}1 {Ö}{{\"O}}1 {Ü}{{\"U}}1
  {â}{{\^a}}1 {ê}{{\^e}}1 {î}{{\^i}}1 {ô}{{\^o}}1 {û}{{\^u}}1
  {Â}{{\^A}}1 {Ê}{{\^E}}1 {Î}{{\^I}}1 {Ô}{{\^O}}1 {Û}{{\^U}}1
  {œ}{{\oe}}1 {Œ}{{\OE}}1 {æ}{{\ae}}1 {Æ}{{\AE}}1 {ß}{{\ss}}1
  {ű}{{\H{u}}}1 {Ű}{{\H{U}}}1 {ő}{{\H{o}}}1 {Ő}{{\H{O}}}1
  {ç}{{\c c}}1 {Ç}{{\c C}}1 {ø}{{\o}}1 {å}{{\r a}}1 {Å}{{\r A}}1
  {€}{{\euro}}1 {£}{{\pounds}}1 {«}{{\guillemotleft}}1
  {»}{{\guillemotright}}1 {ñ}{{\~n}}1 {Ñ}{{\~N}}1 {¿}{{?`}}1
}
\newcommand{\tablemat}{~}

%%%%%%%%%%%%%%%%% TITRE %%%%%%%%%%%%%%%%
\newcommand{\intitule}{Projet 1 : Algorithmes de tri}
\newcommand{\GrNbr}{1}
\newcommand{\PrenomUN}{Martin}
\newcommand{\NomUN}{RANDAXHE}
\newcommand{\PrenomDEUX}{Cyril}
\newcommand{\NomDEUX}{RUSSE}
\renewcommand{\tablemat}{\tableofcontents}

\title{\textbf{INFO0250 - Programmation avancée} \\\intitule}
\author{Groupe \GrNbr : \PrenomUN~\textsc{\NomUN}, \PrenomDEUX~\textsc{\NomDEUX}}
\date{}
\begin{document}
\maketitle
\newpage
\tablemat
\newpage

%%%%%%%%%%%%%%%%%%%%%%%%%%%%%%%%%%%%%%%%%%%%%%%%
\begin{document}
    \section{\textbf{Algorithmes vus au cours}}
    Dans cette section, nous allons analyser les différents algorithmes 
    ayant été vus lors du cours théorique et implémenté dans le cadre de ce projet.
    Pour ce faire, nous allons présenter les résultats des calculs empiriques des temps 
    d'exécution des 3 algorithmes concernés. Par la suite, nous comparerons l'évolution de ceux-ci 
    par rapport aux complexités théoriques en fonction de la taille du tableau. Et finalement y apporter 
    nos analyses vis à vis de l'ordre relatif de ces différents algorithmes.

    \subsection{Résultats empiriques de temps d'exécution}
    Voici, présenté dans la table ci-dessous, les valeurs de temps d'exécution des 3 algorithmes de tri 
    vus au cours: InsertionSort, QuickSort et HeapSort.(Algorithmes respectivement 
    implémentés dans les fichiers InsertionSort.c, QuickSort.c et HeapSort.C)
    Ces valeurs sont basées sur une moyenne de 10 temps d'execution pour des tableaux 
    de taille $10^3$, $10^4$ et $10^5$.

    \begin{figure}[h]
        \centering
        \begin{tabular}{|l|c|c|c|c|c|c|}
            \cline{1-7}
            Type de tableau&\multicolumn{3}{c|}{aléatoire}&\multicolumn{3}{c|}{croissant}\\
            \cline{1-7}
            Taille&$10^3$&$10^4$&$10^5$&$10^3$&$10^4$&$10^5$\\
            \cline{1-7}
            InsertionSort&0.001939&0.06914&6.65226&0.000017&0.000146&0.001397\\
            QuickSort&0.000377&0.001536&0.044997&0.011423&0.33125&31.01124\\
            HeapSort&0.000291&0.003594&0.009951&0.00008&0.000463&0.003292\\
            PlaceSort&0.009042&0.479786&41.534067&0.00707&0.230783&22.825008\\
            RecSort&0.000813&0.002564&0.031851&0.004871&0.0127861&11.120391\\
            \cline{1-7}
        \end{tabular}
        \caption{Résultats empiriques de temps d'exécution}
    \end{figure}

    \subsection{Analyse des tests en fonction des complexités théoriques des différents algorithmes}
    \begin{enumerate}
        \item InsertionSort:
        \begin{itemize}
            \item Croissant : 
            Nous retrouvons entre $10^3$ et $10^4$ un facteur 8,59 et entre 
            $10^4$ et $10^5$ un facteur 9.57. En moyenne, cela nous fait $9.08\simeq 10$.
            La complexité $\Theta(n)$ est vérifiée.
            \item Aléatoire : 
            Nous retrouvons entre $10^3$ et $10^4$ un facteur 35.66 et entre 
            $10^4$ et $10^5$ un facteur 96.21. En moyenne, cela nous fait 65.93. Nous 
            nous rapprochons bel et bien du facteur $10^2$ attendu, d'autant plus notable 
            entre les tableaux $10^4$ et $10^5$.
            La complexité $\Theta(n^2)$ est vérifiée.
        \end{itemize}
        \item QuickSort :
        \begin{itemize}
            \item Croissant : 
            Nous retrouvons entre $10^3$ et $10^4$ un facteur 93.62 et entre 
            $10^4$ et $10^5$ un facteur 29. En moyenne, cela nous fait 61.31. Nous 
            nous rapprochons bel et bien du facteur $10^2$ attendu, d'autant plus notable 
            entre les tableaux $10^4$ et $10^5$.
            La complexité $\Theta(n^2)$, étant donné que nous sommes dans le pire 
            cas pour le QuickSort qui va toujours avoir sa valeur de pivot qui 
            ne permettra pas de "diviser pour mieux régner", est vérifiée.

            \item Aléatoire : 
            Nous retrouvons entre $10^3$ et $10^4$ un facteur 4.07 et entre 
            $10^4$ et $10^5$ un facteur 29.29. En moyenne, cela nous fait 16.68. Nous 
            nous rapprochons bel et bien du facteur $10log(10)$ attendu, d'autant plus notable 
            entre les tableaux $10^4$ et $10^5$.
            La complexité $\Theta(n*log(n))$ est vérifiée.
        \end{itemize}
        \item HeapSort :
        \begin{itemize}
            \item Croissant : 
            Nous retrouvons entre $10^3$ et $10^4$ un facteur 5.79 et entre 
            $10^4$ et $10^5$ un facteur 7.11. En moyenne, cela nous fait 6.45.
            Théoriquement la complexité du heapsort est $\Theta(n*log(n))$, 
            nous nous en rapprochons dans ce cas-ci, les résultats sont mêmes meilleurs que ceux 
            attendus.
            \item Aléatoire : 
            Nous retrouvons entre $10^3$ et $10^4$ un facteur 12.35 et entre 
            $10^4$ et $10^5$ un facteur 2.77. En moyenne, cela nous fait 7.56.
            La complexité $\Theta(n*log(n))$ est vérifiée.
            Cet algorithme est de loin le plus performant quelque soit le cas, 
            il respecte sa complexité théorique de $\Theta(n*log(n))$ quelque soit 
            le cas et est même plus efficace que les autres algorithmes ayant 
            cette même complexité.
        \end{itemize}
    \end{enumerate}

    \subsection{Ordre relatif entre les différents algorithme}
    \begin{itemize}
        \item Croissant : Pour le cas croissant il n'est pas difficile de donner 
        un ordre entre ces algorithme car l'InsertionSort se trouve dans son meilleur cas qui est $\Theta(n)$ 
        qui est le meilleur vu que les 2 autres sont dans leur meilleur cas $\Theta(n*log(n))$. Le cas croissant est 
        le pire cas pour le QuickSort car sa valeur de pivot ne sera jamais à un endroit opportun pour séparé le 
        tableau en plus petits tableaux. Sa complexité est dès lors $\Theta(n^2)$ et le HeapSort est quant à lui toujours 
        $\Theta(n*log(n))$. 
        \\Nous avons donc dans l'ordre (Ordre confirmé par nos tests pour $10^5)$ : \\InsertionSort(0.001397s)$\rightarrow$ HeapSort(0.003292s)$\rightarrow$ QuickSort(31.01124s).    
        
        \item Aléatoire : Pour les 3 algorithmes, nous nous trouvons 
        dans les cas moyens. Nous avons donc l'InsertionSort qui est le 
        moins efficace avec une complexité théorique $\Theta(n^2)$.Ensuite, 
        nous devons départager le QuickSort et le HeapSort qui sont dans 
        ce cas-ci, tous les deux $\Theta(n*log(n))$. Par nos observations, 
        nous pouvons déterminer que le HeapSort est légerement plus efficace. 
        En effet, nous savons que ces 2 algorithmes sont les meilleurs algorithmes 
        possible en terme d'algorithme de tri comparatif. En général, une préférence pour le quicksort qui se focalise un peu plus sur certaines parties 
        du tableau tandis que le heapsort va "chercher" des valeurs un peu partout 
        ce qui peu être moins avantageux d'un point de vue accès mémoire. 
        Mais pour une analyse purement théorique au niveau des performances temporelles, le HeapSort 
        est meilleur.
        \\Nous avons donc dans l'ordre (Ordre confirmé par nos tests pour $10^5)$ : \\HeapSort(0.009951s)$\rightarrow$ QuickSort(0.044997s)$\rightarrow$ InsertionSort(6.65226s).

    \end{itemize}
    \section{\textbf{PlaceSort}}
    \subsection{Pseudo-code}
    Voici le pseudo-code de la fonction Place. Celle-ci utilise A, le 
    tableau à trier A[1...A.length]. La fonction swap, permute les valeurs du tableau 
    qui lui sont données en argument. Nous allons utiliser m comme valeur 
    permettant de définir la position de la valeur sur laquelle nous sommes actuellement 
    qui sera notre à l'indice i.
    \begin{lstlisting}
        PLACE(A)
            i=1
            while(i<=A.length)
                m=1
                for j=1 to A.length
                    if A[i]>A[j]
                        m = m + 1
                if m!=i
                    if A[i]==A[m]
                        while (A[i]==A[m])
                            m = m + 1
                        swap(A[i], A[m])
                        i = i + 1
                    else
                        swap(A[i], A[m])
                else
                    i = i + 1
    \end{lstlisting}
    La fonction Place(A) va placer l'élément A[i]($1\leq i\leq $ A.length) à la position 
    qu'il occuperait dans le tableau A s'il était trié. Soit m, l'emplacement 
    correct de la valeur A[i] alors on permute A[i] et A[m]. On recommence en boucle 
    avec la nouvelle valeur se trouvant en A[i] jusqu'à ce qu'il n'y ait pas de permutation possible 
    où l'on va dès lors incrémenter i afin de trouver une nouvelle valeur qui n'est pas 
    à sa place. Nous prenons également en compte le cas particulier où il existe 
    plusieurs fois la même valeur dans le tableau à trier. Pour ce faire, si l'emplacement correct 
    de la valeur est déjà occupé par un élément de cette même valeur, nous essayons avec la case d'après 
    et ceci en boucle jusqu'à trouver un des cases suivantes ayant une valeur 
    différente.



\end{document}